\documentclass[]{scrartcl}

\RequirePackage[intlimits]{amsmath}
\RequirePackage{amsfonts,amssymb,amscd,amsthm,xspace}

\usepackage[centerlast,small,sc]{caption}
\setlength{\captionmargin}{20pt}

\usepackage[square, numbers, comma, sort&compress]{natbib} 
\usepackage{verbatim} 
\usepackage{graphbox}
\usepackage{vector}  
\usepackage{wasysym}
\usepackage[ruled]{algorithm2e}
\usepackage{enumerate}
\usepackage{mathtools}
\usepackage{hyperref}
%opening
\title{COVID-Predictor}
\subtitle{Derivation and Extension of the Population Model}
\author{Robin Kramer, Tom Kuchler, Lorin M{\"u}hlebach, \\Cedric M{\"u}nger, Mar{\'i}a Ruiz
	Martínez, Yannick Schaffner, \\Dominik Schulte, Kamil Sklodowski
}
\newcommand{\eps}{\varepsilon}
\newcommand{\veps}{\epsilon}
\newcommand{\vphi}{\varphi}
\newcommand{\rnd}[1]{\mathsf{#1} }
%----------------------------------------------
%Commands Robin
%----------------------------------------------
\RequirePackage{bm}
\RequirePackage{bbm}
\RequirePackage{mathrsfs}
\RequirePackage{commath}

\makeatletter 
\def \moverlay{\mathpalette \mov@rlay}
\def \mov@rlay#1#2{\leavevmode\vtop{
		\baselineskip \z@skip \lineskiplimit - \maxdimen
		\ialign{\hfil$\m@th#1##$\hfil\cr#2\cr\cr}
}}
\newcommand{\charfusion}[3][\mathord]{
	#1{\ifx#1\mathop\vphantom{#2}\fi
		\mathpalette\mov@rlay{#2\cr#3}
	}
	\ifx#1 \mathop \expandafter \displaylimits\fi }
\makeatother
%\newcommand{\cupdot}{\mathbin{\mathaccent \cup \cdot }}
\newcommand{\cupdot}{\charfusion[\mathbin]{\cup}{\cdot}}
\newcommand{\bigcupdot}{\charfusion[\mathop]{\bigcup}{\cdot}}



\newcommand{\Def}{\textbf{Def: }}
\newcommand{\Prop}{\textbf{Prop: }}
\newcommand{\Thm}{\textbf{Theorem: }}
\newcommand{\Prf}{\textbf{Proof: }}

\DeclareMathOperator{\Span}{span}
\DeclareMathOperator{\scd}{scd} %set of all common divisors

\newcommand{\R}{\bbR} %reals
%\newcommand{\C}{\bbC} %Complex
\newcommand{\Z}{\bbZ} %integers
\newcommand{\N}{\bbN} %positive integers
\newcommand{\Q}{\bbQ} %Fractional numbers
\newcommand{\LL}{\bbL} %linear Operator
\newcommand{\horbar}{\text{---} } %usefull for matrix of rowvectors 

\newcommand{\floor}[1]{\left\lfloor #1 \right\rfloor}
\newcommand{\ceil}[1]{\left\lceil #1 \right\rceil}
%------------------------
%Vectorspaces and Sets
%------------------------
\newcommand{\sprod}[2]{\ensuremath{ \left\langle #1, \, #2 \right\rangle} } %Scalar product
%\newcommand{\abs}[1]{\ensuremath{\left|#1\right| }}
%\newcommand{\norm}[1]{\ensuremath{\left\Vert #1 \right\Vert}} %the norm
\newcommand{\normL}[2]{\ensuremath{ \norm{#2}_{#1} }} %Specific Norm
\newcommand{\dist}[3][]{
	\ifthenelse
	{\isempty{#2} \AND \isempty{#3}}%Both Args Empty
	{
		\ifthenelse
			{\isempty{#1}}
			{\rho} 
			{\rho_{#1}}
	}
	{
		\ifthenelse
		{\isempty{#2} \OR \isempty{#3} }
		{\PackageError{Usage of dist}{One nonempty arguement detected} } %One Arg nonempty
		{ %Both Args Nonempty
			\ifthenelse
			{\isempty{#1}}
			{\rho(#2,#3)} 
			{\rho_{#1}(#2,#3)}
		}
	}
}
\newcommand{\bigSet}[2]{\ensuremath{ \left\{ {#1}_k \right\}_{k \in #2} } }%Set with indexed elements

\DeclareMathOperator*{\argmax}{argmax}
\DeclareMathOperator*{\argmin}{argmin}
\DeclareMathOperator*{\esssup}{ess\,sup}
\DeclareMathOperator*{\essinf}{ess\,inf}
\RequirePackage{bbm}
\newcommand{\indicFkt}[2]{\ensuremath{ \mathbbm{1}_{#1}\left(#2\right) }}
\newcommand{\indicSet}[1]{\ensuremath{ \mathbbm{1}_{\left\{ #1 \right\}} }}

\newcommand{\almostE}{\text{-a.e. }}
\newcommand{\almostA}{\text{-a.a. }}
\newcommand{\almostS}{\text{-a.s. }}
\newcommand{\singular}{\bot}

\DeclareMathOperator{\diam}{diam}
\newcommand{\closure}[1]{\mathrm{cl}\del{#1}}

\DeclareMathOperator{\identity}{I}
\newcommand{\ident}[1]{\identity_{#1}}
\newcommand{\ones}{\ensurevect{1}{}}

\DeclareMathOperator{\supp}{supp} %Support set

\RequirePackage{xifthen}
\RequirePackage{bm}
\newcommand{\ensurevect}[2]{ %Makes the vector boldfaced if no indices is given (second Arg empty)
	\ensuremath{ 
		\ifthenelse{\isempty{#2} \or \equal{#2}{,}} %or part due to bad implementation necessary, 
		%should have used optional arguments \cmd[optArg] from the beginning
		{\bm{#1} }
		{#1_{#2} }
	}
}

\newcommand{\condSet}[2]{\ensuremath{\left\{#1\vphantom{#2}\right|\left. \vphantom{#1}#2 \right\} }}
\newcommand{\compl}[1]{{#1}^{\mathsf{c}}}

%Transpose symbol with and w/o brackets
\newcommand{\trps}[1]{\ensuremath{ #1^{\mathsf{T}} }}
\newcommand{\trpsbr}[1]{ \trps{\left(#1 \right)} }
\newcommand{\hermtr}[1]{\ensuremath{ #1^{\mathsf{H}} }}
\newcommand{\hermtrbr}[1]{ \hermtr{\left(#1 \right)} }


%-----------------
% Arrows
% ---------------
\newcommand{\larr}{\ensuremath{\leftarrow}}
\newcommand{\Larr}{\ensuremath{\Leftarrow}}
\newcommand{\rarr}{\ensuremath{\rightarrow}}
\newcommand{\Rarr}{\ensuremath{\Rightarrow}}
\newcommand{\lrarr}{\ensuremath{\leftrightarrow}}
\newcommand{\Lrarr}{\ensuremath{\Leftrightarrow}}

\newcommand{\urarr}{\ensuremath{\nearrow}}
\newcommand{\drarr}{\ensuremath{\searrow}}
\newcommand{\ularr}{\ensuremath{\nwarrow}}
\newcommand{\dlarr}{\ensuremath{\swarrow}}
\newcommand{\goesto}[1]{\ensuremath{\xrightarrow{#1} }}

%--------------------
% Probability Stuff
%--------------------
\newcommand{\EE}[1][]{\mathrm{E_{#1} }} %Expected Value
\DeclareMathOperator{\var}{Var} %variance
\DeclareMathOperator{\cov}{Cov} %covariance
%\newcommand{\prb}[1]{\mathrm{Pr}\sbr{#1} }
\DeclareMathOperator{\HH}{H}
\DeclareMathOperator{\II}{I}
\DeclareMathOperator{\RelEntropy}{D}
\newcommand{\DD}[2]{\ensuremath{\RelEntropy \left( #1\vphantom{#2}\right\Vert\left. \vphantom{#1}#2 \right) }}
\newcommand{\HR}[1]{\HH_{#1}} %Renyi Entropy

\newcommand{\unif}[1]{\ensuremath{ \calU \left[ #1 \right] }}
\newcommand{\expDist}[1]{\ensuremath{ \mathsf{EXP}\left( #1 \right) }}
\newcommand{\gaussian}[2]{\calN\left(#1,#2\right)}

\newcommand{\prbFkt}[3]{\ensuremath{ #3_{#1}
		\ifthenelse{\isempty{#2}}{}
		{\left( #2\right)} 
}}
\newcommand{\prbFktCond}[5]{\ensuremath{ 
		#5_{
			\left. \vphantom{#2}#1\right|#2
		}
		\ifthenelse{\isempty{#3} \or \isempty{#4}}
		{} %at least one argument empty, do nothing
		{
			\left( \vphantom{#4}#3\right|\left. \vphantom{#3}#4\right) 
		}
}}
\newcommand{\pdf}[2]{\prbFkt{#1}{#2}{f}}
\newcommand{\pdfCond}[4]{\prbFktCond{#1}{#2}{#3}{#4}{f}}
\newcommand{\cdf}[2]{\prbFkt{#1}{#2}{F}}
\newcommand{\cdfCond}[4]{\prbFktCond{#1}{#2}{#3}{#4}{F}}
\newcommand{\pmf}[2]{\prbFkt{#1}{#2}{p}}
\newcommand{\pmfCond}[4]{\prbFktCond{#1}{#2}{#3}{#4}{p}}
\newcommand{\prb}[1]
{\ensuremath{
		\mathrm{Pr}{\left[ #1\right]} 
}}
\newcommand{\prbCond}[2]
{
	\mathrm{Pr}\left[ \vphantom{#2}#1\right|\left. \vphantom{#1}#2\right]
}


\newcommand{ \calA}{\ensuremath{\mathcal A}}
\newcommand{ \calB}{\ensuremath{\mathcal B}}
\newcommand{ \calC}{\ensuremath{\mathcal C}}
\newcommand{ \calD}{\ensuremath{\mathcal D}}
\newcommand{ \calE}{\ensuremath{\mathcal E}}
\newcommand{ \calF}{\ensuremath{\mathcal F}}
\newcommand{ \calG}{\ensuremath{\mathcal G}}
\newcommand{ \calH}{\ensuremath{\mathcal H}}
\newcommand{ \calI}{\ensuremath{\mathcal I}}
\newcommand{ \calJ}{\ensuremath{\mathcal J}}
\newcommand{ \calK}{\ensuremath{\mathcal K}}
\newcommand{ \calL}{\ensuremath{\mathcal L}}
\newcommand{ \calM}{\ensuremath{\mathcal M}}
\newcommand{ \calN}{\ensuremath{\mathcal N}}
\newcommand{ \calO}{\ensuremath{\mathcal O}}
\newcommand{ \calP}{\ensuremath{\mathcal P}}
\newcommand{ \calQ}{\ensuremath{\mathcal Q}}
\newcommand{ \calR}{\ensuremath{\mathcal R}}
\newcommand{ \calS}{\ensuremath{\mathcal S}}
\newcommand{ \calT}{\ensuremath{\mathcal T}}
\newcommand{ \calU}{\ensuremath{\mathcal U}}
\newcommand{ \calV}{\ensuremath{\mathcal V}}
\newcommand{ \calW}{\ensuremath{\mathcal W}}
\newcommand{ \calX}{\ensuremath{\mathcal X}}
\newcommand{ \calY}{\ensuremath{\mathcal Y}}
\newcommand{ \calZ}{\ensuremath{\mathcal Z}}

\newcommand{ \bbA}{\ensuremath{\mathbb A}}
\newcommand{ \bbB}{\ensuremath{\mathbb B}}
\newcommand{ \bbC}{\ensuremath{\mathbb C}}
\newcommand{ \bbD}{\ensuremath{\mathbb D}}
\newcommand{ \bbE}{\ensuremath{\mathbb E}}
\newcommand{ \bbF}{\ensuremath{\mathbb F}}
\newcommand{ \bbG}{\ensuremath{\mathbb G}}
\newcommand{ \bbH}{\ensuremath{\mathbb H}}
\newcommand{ \bbI}{\ensuremath{\mathbb I}}
\newcommand{ \bbJ}{\ensuremath{\mathbb J}}
\newcommand{ \bbK}{\ensuremath{\mathbb K}}
\newcommand{ \bbL}{\ensuremath{\mathbb L}}
\newcommand{ \bbM}{\ensuremath{\mathbb M}}
\newcommand{ \bbN}{\ensuremath{\mathbb N}}
\newcommand{ \bbO}{\ensuremath{\mathbb O}}
\newcommand{ \bbP}{\ensuremath{\mathbb P}}
\newcommand{ \bbQ}{\ensuremath{\mathbb Q}}
\newcommand{ \bbR}{\ensuremath{\mathbb R}}
\newcommand{ \bbS}{\ensuremath{\mathbb S}}
\newcommand{ \bbT}{\ensuremath{\mathbb T}}
\newcommand{ \bbU}{\ensuremath{\mathbb U}}
\newcommand{ \bbV}{\ensuremath{\mathbb V}}
\newcommand{ \bbW}{\ensuremath{\mathbb W}}
\newcommand{ \bbX}{\ensuremath{\mathbb X}}
\newcommand{ \bbY}{\ensuremath{\mathbb Y}}
\newcommand{ \bbZ}{\ensuremath{\mathbb Z}}


\newcommand{ \bfA}{\ensuremath{\mathbf A}}
\newcommand{ \bfB}{\ensuremath{\mathbf B}}
\newcommand{ \bfC}{\ensuremath{\mathbf C}}
\newcommand{ \bfD}{\ensuremath{\mathbf D}}
\newcommand{ \bfE}{\ensuremath{\mathbf E}}
\newcommand{ \bfF}{\ensuremath{\mathbf F}}
\newcommand{ \bfG}{\ensuremath{\mathbf G}}
\newcommand{ \bfH}{\ensuremath{\mathbf H}}
\newcommand{ \bfI}{\ensuremath{\mathbf I}}
\newcommand{ \bfJ}{\ensuremath{\mathbf J}}
\newcommand{ \bfK}{\ensuremath{\mathbf K}}
\newcommand{ \bfL}{\ensuremath{\mathbf L}}
\newcommand{ \bfM}{\ensuremath{\mathbf M}}
\newcommand{ \bfN}{\ensuremath{\mathbf N}}
\newcommand{ \bfO}{\ensuremath{\mathbf O}}
\newcommand{ \bfP}{\ensuremath{\mathbf P}}
\newcommand{ \bfQ}{\ensuremath{\mathbf Q}}
\newcommand{ \bfR}{\ensuremath{\mathbf R}}
\newcommand{ \bfS}{\ensuremath{\mathbf S}}
\newcommand{ \bfT}{\ensuremath{\mathbf T}}
\newcommand{ \bfU}{\ensuremath{\mathbf U}}
\newcommand{ \bfV}{\ensuremath{\mathbf V}}
\newcommand{ \bfW}{\ensuremath{\mathbf W}}
\newcommand{ \bfX}{\ensuremath{\mathbf X}}
\newcommand{ \bfY}{\ensuremath{\mathbf Y}}
\newcommand{ \bfZ}{\ensuremath{\mathbf Z}}

\newcommand{ \setA}{\ensuremath{\mathcal A}}
\newcommand{ \setB}{\ensuremath{\mathcal B}}
\newcommand{ \setC}{\ensuremath{\mathcal C}}
\newcommand{ \setD}{\ensuremath{\mathcal D}}
\newcommand{ \setE}{\ensuremath{\mathcal E}}
\newcommand{ \setF}{\ensuremath{\mathcal F}}
\newcommand{ \setG}{\ensuremath{\mathcal G}}
\newcommand{ \setH}{\ensuremath{\mathcal H}}
\newcommand{ \setI}{\ensuremath{\mathcal I}}
\newcommand{ \setJ}{\ensuremath{\mathcal J}}
\newcommand{ \setK}{\ensuremath{\mathcal K}}
\newcommand{ \setL}{\ensuremath{\mathcal L}}
\newcommand{ \setM}{\ensuremath{\mathcal M}}
\newcommand{ \setN}{\ensuremath{\mathcal N}}
\newcommand{ \setO}{\ensuremath{\mathcal O}}
\newcommand{ \setP}{\ensuremath{\mathcal P}}
\newcommand{ \setQ}{\ensuremath{\mathcal Q}}
\newcommand{ \setR}{\ensuremath{\mathcal R}}
\newcommand{ \setS}{\ensuremath{\mathcal S}}
\newcommand{ \setT}{\ensuremath{\mathcal T}}
\newcommand{ \setU}{\ensuremath{\mathcal U}}
\newcommand{ \setV}{\ensuremath{\mathcal V}}
\newcommand{ \setW}{\ensuremath{\mathcal W}}
\newcommand{ \setX}{\ensuremath{\mathcal X}}
\newcommand{ \setY}{\ensuremath{\mathcal Y}}
\newcommand{ \setZ}{\ensuremath{\mathcal Z}}

%Contains most usefull commands
\begin{document}

\maketitle

%\begin{abstract}
%
%\end{abstract}

\section{Generating Households, Implementation Versus-Virus}
Goal: get the age distribution of the household members $\prb{\rnd A_1,...,\rnd A_{\rnd S},\rnd T}$
which largely determines the infection rate within household. Assumptions about the natural interaction \footnote{e.g. family vs. roommates} and occupation \footnote{Influence of infections imported from outside} of the household members has an important impact on how the disease spreads.
The method to create natural households by Ajelli et Al. can be summarized as follows:

\begin{align}
  \prb{\rnd T}{}: &\text{Prb of Houshold type }\\
  \prbCond{\rnd S}{\rnd T}{}{}: & \text{Prb of Houshold size given Type}\\
  \prbCond{\rnd C}{\rnd T,\rnd S}:  & \text{Prb of Houshold Head Age class, }\\
  &\text{Is an intermediate step from data-availablility on age in houshold}\notag\\
  \prbCond{\rnd A_1}{\rnd C}: & \text{Prb of Houshold head age}\\
  \prbCond{\rnd A_2,...,\rnd A_s}{s=S}: &\text{Prb of age of other houshold members}
\end{align}

The available data for Italy does however not have a one-to-one  correspondence to Switzerland. 
Thanks to the swiss Federal Statistical Office, a wide variety of relevant data is available:
\begin{itemize}
	\item Number of persons of age $x$ per municipality
	\\ \url{https://www.bfs.admin.ch/bfs/de/home/statistiken/bevoelkerung/stand-entwicklung/bevoelkerung.assetdetail.9635941.html}
	\item Number of households with $n$ members $n\in \{1,2,...,5,6+\}$ per municipality\\ \url{https://www.bfs.admin.ch/bfs/de/home/statistiken/bevoelkerung/stand-entwicklung/haushalte.assetdetail.9787080.html}
	\item Prevalence of different forms of households (Familienbericht G2.1)
	\item Household type per person and per household (Familienbericht G2.2)
	\item Parent-childern households, per age of youngest child (over/under 25) and number of parents (Familienbericht G2.3)
	\item Proportion of parents in same household according to age categories 0-3, 4-12,13-17,18-14 (Familienbericht G2.4)
	\item Distribution of number of children below 25 in household (Familienbericht G2.8)
	\item Distribution of age difference in couples living in same household (Familienbericht G3.2)
\end{itemize}
The "Familienbericht" is available under \url{https://www.bfs.admin.ch/bfs/de/home/statistiken/kataloge-datenbanken/publikationen.assetdetail.2347880.html} and the relevant datasets can be found in the "Anhang".
From this, we propose the following algorithm for generating realistic population and household data:


\subsection{Used Approximation}
From the available data for Switzerland we decided to use for version 0.1
\begin{itemize}
	\item the distribution of person/household per municipality,
	\item the age distribution per municipality.
\end{itemize}
Then we divided the citizens in the three age-classes 0-18, 18-65, 65+ and proceeded as follows:
\begin{enumerate}
	\item generate per municipality the correct number of households w. occupants 
		(1,2,3,4,5,6+) members)
	\item assign first member according to local age distribution for 18+
	\item assign rest of members according to the resulting marginal distribution after excluding the first member
	(6+ households get only 6 members assigned)
\end{enumerate}
	
\section{Proposed Algorithmic Refinements}
	In order to refine the approach developed and implemented during VersusVirus, we can propose some improvements, starting with increasing the variety of age categories.
	A reasonable granularity of the age categories, considering the different natural behaviour of each age group, different effects by quarantine measures, and available data is shown in table \ref{tab:Age}.
	\begin{table}
		\centering
		\begin{tabular}{c| c r}
			Cat&Age&Likeli Occupation\\
			\hline
			A& 0-5 & Preschool \\
			B&6-17 & School/Apprenticeship \\
			C&18-24 & Working\footnote{incl. apprenticeship}/Student  \\
			D&25-49 & Working\\
			E&50-65 & Working\\
			F&65+ &retired, high risk
		\end{tabular}
		\caption{Proposed classification based on age.}
		\label{tab:Age}
	\end{table}
	\begin{table}
		\centering
		\begin{tabular}{c|r r}
			cat&\#Pers&Constraints
			\\
			\hline
			I&1&adult\\
			II-C&2&Couple\\
			II-P&2&Parent-Child\\
			III-C&3&Couple. 1 Child\\
			III-P&3&Single, 2 Children\\
			IV-C&4&Couple, 2 Children\\
			IV-A&4&All adults\\
			V-C&5&Couple, 3 children\\
			V-A&5&All adults\\
			VI&6+&All adults
		\end{tabular}
		\caption{Proposed simplified household categories}
		\label{tab:HH}
	\end{table}
	Some possible categories are not proposed due to the low prevalence to simplify the household generation.
	The constraints within each household are as follows:
	\begin{itemize}
		\item Adults are in age categories C-F
		\item Children are in age categories A-C
		\item Couples constitute of adults, differing at most by 12 years
		\item Parents and children in the same household have an age-difference between 18 and 40 years
		\item thus adults in household-types\footnote{except II-C} x-C, x-P are in age categories C-E
	\end{itemize}
	
	\subsection{Generating the Households}
	\begin{enumerate}
		\item generate municipality households with correct distribution of member numbers categories
		 \item calculate the ratio $\frac{\text{II-P}}{\text{II-C}}$ and assign households w. 2 members accordingly.
		 \begin{itemize}
		 \item[II-P] single parents living with children: $4.3\%$ of all households, around $61\%$ of all single parents live with one child. 
		\item[II-C]
		 around $25\%$ of all households constitute couples w/o kids
		\end{itemize}
		\item in the same fashion, calculate the split for the household types III-X, IV-X, V-X	
	\end{enumerate}
	\subsection{Populating the Households}
	\begin{enumerate}	
		\item generate population according to municipal age distribution
		\item assign\footnote{assignments happen uniform at random from the unassigned population meeting the criteria. This corresponds to the empirical conditional distribution} for each household of categories II-P, III-P, III-C, IV-P, V-P one adult in age group C-E
		\item assign for III-C, IV-C households spouse meeting the age-constraints
		\item assign correct number of children meeting the age constraints
		\item Form valid couples for category $II-P$
		\item Assign I to remaining adults
		\item Assign households of type VI one adult each
		\item Fill up the households of type VI as good as possible
	\end{enumerate}
	
	\subsection{Constraints Violations}
	The proposed algorithm does not guarantee that all constraints can be met. In particular, the population data is per municipality, but the data on different household forms are country-wide averages. Possible solutions for violations are:
	\begin{itemize}
		\item Try again with different RNG-seed
		\item Implement optimization algorithm
		\item take agent of closest fit instead
		\item Generate new random agent according to population distribution 
	\end{itemize}

\end{document}
