%----------------------------------------------
%Commands Robin
%----------------------------------------------
\RequirePackage{bm}
\RequirePackage{bbm}
\RequirePackage{mathrsfs}
\RequirePackage{commath}

\makeatletter 
\def \moverlay{\mathpalette \mov@rlay}
\def \mov@rlay#1#2{\leavevmode\vtop{
		\baselineskip \z@skip \lineskiplimit - \maxdimen
		\ialign{\hfil$\m@th#1##$\hfil\cr#2\cr\cr}
}}
\newcommand{\charfusion}[3][\mathord]{
	#1{\ifx#1\mathop\vphantom{#2}\fi
		\mathpalette\mov@rlay{#2\cr#3}
	}
	\ifx#1 \mathop \expandafter \displaylimits\fi }
\makeatother
%\newcommand{\cupdot}{\mathbin{\mathaccent \cup \cdot }}
\newcommand{\cupdot}{\charfusion[\mathbin]{\cup}{\cdot}}
\newcommand{\bigcupdot}{\charfusion[\mathop]{\bigcup}{\cdot}}



\newcommand{\Def}{\textbf{Def: }}
\newcommand{\Prop}{\textbf{Prop: }}
\newcommand{\Thm}{\textbf{Theorem: }}
\newcommand{\Prf}{\textbf{Proof: }}

\DeclareMathOperator{\Span}{span}
\DeclareMathOperator{\scd}{scd} %set of all common divisors

\newcommand{\R}{\bbR} %reals
%\newcommand{\C}{\bbC} %Complex
\newcommand{\Z}{\bbZ} %integers
\newcommand{\N}{\bbN} %positive integers
\newcommand{\Q}{\bbQ} %Fractional numbers
\newcommand{\LL}{\bbL} %linear Operator
\newcommand{\horbar}{\text{---} } %usefull for matrix of rowvectors 

\newcommand{\floor}[1]{\left\lfloor #1 \right\rfloor}
\newcommand{\ceil}[1]{\left\lceil #1 \right\rceil}
%------------------------
%Vectorspaces and Sets
%------------------------
\newcommand{\sprod}[2]{\ensuremath{ \left\langle #1, \, #2 \right\rangle} } %Scalar product
%\newcommand{\abs}[1]{\ensuremath{\left|#1\right| }}
%\newcommand{\norm}[1]{\ensuremath{\left\Vert #1 \right\Vert}} %the norm
\newcommand{\normL}[2]{\ensuremath{ \norm{#2}_{#1} }} %Specific Norm
\newcommand{\dist}[3][]{
	\ifthenelse
	{\isempty{#2} \AND \isempty{#3}}%Both Args Empty
	{
		\ifthenelse
			{\isempty{#1}}
			{\rho} 
			{\rho_{#1}}
	}
	{
		\ifthenelse
		{\isempty{#2} \OR \isempty{#3} }
		{\PackageError{Usage of dist}{One nonempty arguement detected} } %One Arg nonempty
		{ %Both Args Nonempty
			\ifthenelse
			{\isempty{#1}}
			{\rho(#2,#3)} 
			{\rho_{#1}(#2,#3)}
		}
	}
}
\newcommand{\bigSet}[2]{\ensuremath{ \left\{ {#1}_k \right\}_{k \in #2} } }%Set with indexed elements

\DeclareMathOperator*{\argmax}{argmax}
\DeclareMathOperator*{\argmin}{argmin}
\DeclareMathOperator*{\esssup}{ess\,sup}
\DeclareMathOperator*{\essinf}{ess\,inf}
\RequirePackage{bbm}
\newcommand{\indicFkt}[2]{\ensuremath{ \mathbbm{1}_{#1}\left(#2\right) }}
\newcommand{\indicSet}[1]{\ensuremath{ \mathbbm{1}_{\left\{ #1 \right\}} }}

\newcommand{\almostE}{\text{-a.e. }}
\newcommand{\almostA}{\text{-a.a. }}
\newcommand{\almostS}{\text{-a.s. }}
\newcommand{\singular}{\bot}

\DeclareMathOperator{\diam}{diam}
\newcommand{\closure}[1]{\mathrm{cl}\del{#1}}

\DeclareMathOperator{\identity}{I}
\newcommand{\ident}[1]{\identity_{#1}}
\newcommand{\ones}{\ensurevect{1}{}}

\DeclareMathOperator{\supp}{supp} %Support set

\RequirePackage{xifthen}
\RequirePackage{bm}
\newcommand{\ensurevect}[2]{ %Makes the vector boldfaced if no indices is given (second Arg empty)
	\ensuremath{ 
		\ifthenelse{\isempty{#2} \or \equal{#2}{,}} %or part due to bad implementation necessary, 
		%should have used optional arguments \cmd[optArg] from the beginning
		{\bm{#1} }
		{#1_{#2} }
	}
}

\newcommand{\condSet}[2]{\ensuremath{\left\{#1\vphantom{#2}\right|\left. \vphantom{#1}#2 \right\} }}
\newcommand{\compl}[1]{{#1}^{\mathsf{c}}}

%Transpose symbol with and w/o brackets
\newcommand{\trps}[1]{\ensuremath{ #1^{\mathsf{T}} }}
\newcommand{\trpsbr}[1]{ \trps{\left(#1 \right)} }
\newcommand{\hermtr}[1]{\ensuremath{ #1^{\mathsf{H}} }}
\newcommand{\hermtrbr}[1]{ \hermtr{\left(#1 \right)} }


%-----------------
% Arrows
% ---------------
\newcommand{\larr}{\ensuremath{\leftarrow}}
\newcommand{\Larr}{\ensuremath{\Leftarrow}}
\newcommand{\rarr}{\ensuremath{\rightarrow}}
\newcommand{\Rarr}{\ensuremath{\Rightarrow}}
\newcommand{\lrarr}{\ensuremath{\leftrightarrow}}
\newcommand{\Lrarr}{\ensuremath{\Leftrightarrow}}

\newcommand{\urarr}{\ensuremath{\nearrow}}
\newcommand{\drarr}{\ensuremath{\searrow}}
\newcommand{\ularr}{\ensuremath{\nwarrow}}
\newcommand{\dlarr}{\ensuremath{\swarrow}}
\newcommand{\goesto}[1]{\ensuremath{\xrightarrow{#1} }}

%--------------------
% Probability Stuff
%--------------------
\newcommand{\EE}[1][]{\mathrm{E_{#1} }} %Expected Value
\DeclareMathOperator{\var}{Var} %variance
\DeclareMathOperator{\cov}{Cov} %covariance
%\newcommand{\prb}[1]{\mathrm{Pr}\sbr{#1} }
\DeclareMathOperator{\HH}{H}
\DeclareMathOperator{\II}{I}
\DeclareMathOperator{\RelEntropy}{D}
\newcommand{\DD}[2]{\ensuremath{\RelEntropy \left( #1\vphantom{#2}\right\Vert\left. \vphantom{#1}#2 \right) }}
\newcommand{\HR}[1]{\HH_{#1}} %Renyi Entropy

\newcommand{\unif}[1]{\ensuremath{ \calU \left[ #1 \right] }}
\newcommand{\expDist}[1]{\ensuremath{ \mathsf{EXP}\left( #1 \right) }}
\newcommand{\gaussian}[2]{\calN\left(#1,#2\right)}

\newcommand{\prbFkt}[3]{\ensuremath{ #3_{#1}
		\ifthenelse{\isempty{#2}}{}
		{\left( #2\right)} 
}}
\newcommand{\prbFktCond}[5]{\ensuremath{ 
		#5_{
			\left. \vphantom{#2}#1\right|#2
		}
		\ifthenelse{\isempty{#3} \or \isempty{#4}}
		{} %at least one argument empty, do nothing
		{
			\left( \vphantom{#4}#3\right|\left. \vphantom{#3}#4\right) 
		}
}}
\newcommand{\pdf}[2]{\prbFkt{#1}{#2}{f}}
\newcommand{\pdfCond}[4]{\prbFktCond{#1}{#2}{#3}{#4}{f}}
\newcommand{\cdf}[2]{\prbFkt{#1}{#2}{F}}
\newcommand{\cdfCond}[4]{\prbFktCond{#1}{#2}{#3}{#4}{F}}
\newcommand{\pmf}[2]{\prbFkt{#1}{#2}{p}}
\newcommand{\pmfCond}[4]{\prbFktCond{#1}{#2}{#3}{#4}{p}}
\newcommand{\prb}[1]
{\ensuremath{
		\mathrm{Pr}{\left[ #1\right]} 
}}
\newcommand{\prbCond}[2]
{
	\mathrm{Pr}\left[ \vphantom{#2}#1\right|\left. \vphantom{#1}#2\right]
}


\newcommand{ \calA}{\ensuremath{\mathcal A}}
\newcommand{ \calB}{\ensuremath{\mathcal B}}
\newcommand{ \calC}{\ensuremath{\mathcal C}}
\newcommand{ \calD}{\ensuremath{\mathcal D}}
\newcommand{ \calE}{\ensuremath{\mathcal E}}
\newcommand{ \calF}{\ensuremath{\mathcal F}}
\newcommand{ \calG}{\ensuremath{\mathcal G}}
\newcommand{ \calH}{\ensuremath{\mathcal H}}
\newcommand{ \calI}{\ensuremath{\mathcal I}}
\newcommand{ \calJ}{\ensuremath{\mathcal J}}
\newcommand{ \calK}{\ensuremath{\mathcal K}}
\newcommand{ \calL}{\ensuremath{\mathcal L}}
\newcommand{ \calM}{\ensuremath{\mathcal M}}
\newcommand{ \calN}{\ensuremath{\mathcal N}}
\newcommand{ \calO}{\ensuremath{\mathcal O}}
\newcommand{ \calP}{\ensuremath{\mathcal P}}
\newcommand{ \calQ}{\ensuremath{\mathcal Q}}
\newcommand{ \calR}{\ensuremath{\mathcal R}}
\newcommand{ \calS}{\ensuremath{\mathcal S}}
\newcommand{ \calT}{\ensuremath{\mathcal T}}
\newcommand{ \calU}{\ensuremath{\mathcal U}}
\newcommand{ \calV}{\ensuremath{\mathcal V}}
\newcommand{ \calW}{\ensuremath{\mathcal W}}
\newcommand{ \calX}{\ensuremath{\mathcal X}}
\newcommand{ \calY}{\ensuremath{\mathcal Y}}
\newcommand{ \calZ}{\ensuremath{\mathcal Z}}

\newcommand{ \bbA}{\ensuremath{\mathbb A}}
\newcommand{ \bbB}{\ensuremath{\mathbb B}}
\newcommand{ \bbC}{\ensuremath{\mathbb C}}
\newcommand{ \bbD}{\ensuremath{\mathbb D}}
\newcommand{ \bbE}{\ensuremath{\mathbb E}}
\newcommand{ \bbF}{\ensuremath{\mathbb F}}
\newcommand{ \bbG}{\ensuremath{\mathbb G}}
\newcommand{ \bbH}{\ensuremath{\mathbb H}}
\newcommand{ \bbI}{\ensuremath{\mathbb I}}
\newcommand{ \bbJ}{\ensuremath{\mathbb J}}
\newcommand{ \bbK}{\ensuremath{\mathbb K}}
\newcommand{ \bbL}{\ensuremath{\mathbb L}}
\newcommand{ \bbM}{\ensuremath{\mathbb M}}
\newcommand{ \bbN}{\ensuremath{\mathbb N}}
\newcommand{ \bbO}{\ensuremath{\mathbb O}}
\newcommand{ \bbP}{\ensuremath{\mathbb P}}
\newcommand{ \bbQ}{\ensuremath{\mathbb Q}}
\newcommand{ \bbR}{\ensuremath{\mathbb R}}
\newcommand{ \bbS}{\ensuremath{\mathbb S}}
\newcommand{ \bbT}{\ensuremath{\mathbb T}}
\newcommand{ \bbU}{\ensuremath{\mathbb U}}
\newcommand{ \bbV}{\ensuremath{\mathbb V}}
\newcommand{ \bbW}{\ensuremath{\mathbb W}}
\newcommand{ \bbX}{\ensuremath{\mathbb X}}
\newcommand{ \bbY}{\ensuremath{\mathbb Y}}
\newcommand{ \bbZ}{\ensuremath{\mathbb Z}}


\newcommand{ \bfA}{\ensuremath{\mathbf A}}
\newcommand{ \bfB}{\ensuremath{\mathbf B}}
\newcommand{ \bfC}{\ensuremath{\mathbf C}}
\newcommand{ \bfD}{\ensuremath{\mathbf D}}
\newcommand{ \bfE}{\ensuremath{\mathbf E}}
\newcommand{ \bfF}{\ensuremath{\mathbf F}}
\newcommand{ \bfG}{\ensuremath{\mathbf G}}
\newcommand{ \bfH}{\ensuremath{\mathbf H}}
\newcommand{ \bfI}{\ensuremath{\mathbf I}}
\newcommand{ \bfJ}{\ensuremath{\mathbf J}}
\newcommand{ \bfK}{\ensuremath{\mathbf K}}
\newcommand{ \bfL}{\ensuremath{\mathbf L}}
\newcommand{ \bfM}{\ensuremath{\mathbf M}}
\newcommand{ \bfN}{\ensuremath{\mathbf N}}
\newcommand{ \bfO}{\ensuremath{\mathbf O}}
\newcommand{ \bfP}{\ensuremath{\mathbf P}}
\newcommand{ \bfQ}{\ensuremath{\mathbf Q}}
\newcommand{ \bfR}{\ensuremath{\mathbf R}}
\newcommand{ \bfS}{\ensuremath{\mathbf S}}
\newcommand{ \bfT}{\ensuremath{\mathbf T}}
\newcommand{ \bfU}{\ensuremath{\mathbf U}}
\newcommand{ \bfV}{\ensuremath{\mathbf V}}
\newcommand{ \bfW}{\ensuremath{\mathbf W}}
\newcommand{ \bfX}{\ensuremath{\mathbf X}}
\newcommand{ \bfY}{\ensuremath{\mathbf Y}}
\newcommand{ \bfZ}{\ensuremath{\mathbf Z}}

\newcommand{ \setA}{\ensuremath{\mathcal A}}
\newcommand{ \setB}{\ensuremath{\mathcal B}}
\newcommand{ \setC}{\ensuremath{\mathcal C}}
\newcommand{ \setD}{\ensuremath{\mathcal D}}
\newcommand{ \setE}{\ensuremath{\mathcal E}}
\newcommand{ \setF}{\ensuremath{\mathcal F}}
\newcommand{ \setG}{\ensuremath{\mathcal G}}
\newcommand{ \setH}{\ensuremath{\mathcal H}}
\newcommand{ \setI}{\ensuremath{\mathcal I}}
\newcommand{ \setJ}{\ensuremath{\mathcal J}}
\newcommand{ \setK}{\ensuremath{\mathcal K}}
\newcommand{ \setL}{\ensuremath{\mathcal L}}
\newcommand{ \setM}{\ensuremath{\mathcal M}}
\newcommand{ \setN}{\ensuremath{\mathcal N}}
\newcommand{ \setO}{\ensuremath{\mathcal O}}
\newcommand{ \setP}{\ensuremath{\mathcal P}}
\newcommand{ \setQ}{\ensuremath{\mathcal Q}}
\newcommand{ \setR}{\ensuremath{\mathcal R}}
\newcommand{ \setS}{\ensuremath{\mathcal S}}
\newcommand{ \setT}{\ensuremath{\mathcal T}}
\newcommand{ \setU}{\ensuremath{\mathcal U}}
\newcommand{ \setV}{\ensuremath{\mathcal V}}
\newcommand{ \setW}{\ensuremath{\mathcal W}}
\newcommand{ \setX}{\ensuremath{\mathcal X}}
\newcommand{ \setY}{\ensuremath{\mathcal Y}}
\newcommand{ \setZ}{\ensuremath{\mathcal Z}}

